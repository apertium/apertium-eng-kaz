\documentclass{article}
\usepackage[utf8]{inputenc}
\usepackage{qtree}
\usepackage[bf,small]{caption}
\title{Reorganization of transfer in Apertium-eng-kaz: \\ work performed from June 19 to June 22, 2013\footnote{Mikel L. Forcada's visit has been  funded by the Al-Farabi Kazakh National University through the State program for the attraction of foreign scholars.}}
\author{Mikel L. Forcada,\footnote{Permanent address: Departament de Llenguatges i Sistemes Inform\`{a}tics, Universitat d'Alacant, E-03071 Alacant, Spain.} Aida Sundetova, \\ 
Aizhan Aitkulova, Assem Shormakova \\ 
Information Systems Chair \\ Al-Farabi Kazakh National University, \\ Al-Farabi av., 71, 050040 Almaty, Kazakhstan}
\date{22 June 2013}

\newcommand{\com}[1]{\begin{quote}\begin{sf}#1\end{sf}\end{quote}}
\begin{document}

\maketitle

\section{Introduction}
To be able to deal with the necessary case and possessive marking changes for ``bare'' Kazakh noun phrases (NPs) resulting (mainly) from the translation of English noun phrases, a reorganization of the current 1st-level (\texttt{.t1x}) and 2nd-level (\texttt{.t2x}) transfer rules is currently under way in \texttt{apertium-eng-kaz} (in a subdirectory under the subdirectory \texttt{dev}). Relative \emph{links} equivalent to those using \texttt{link-to}\footnote{Instead of using clumsy \texttt{clip} constructs with \texttt{link-to} references with attributes such as \texttt{side}, \texttt{pos} and \texttt{part} that will eventually be ignored, we have opted for simple \texttt{lit-tag} constructs taking integers as values, as they work just fine. We have commented them for clarity.} in other language pairs, which will be called \emph{references} in this document, are dealt with, as usual, at the 3rd level or \emph{post-chunk} level of transfer (\texttt{.t3x}), with a caveat: as the morphological description of Kazakh available uses optional morphemes for singular (modelled as the absence of the plural \emph{-LAr} morpheme\footnote{An ad-hoc latinization is used in examples, with capitals representing archiphonemes; for instance, \emph{A} can be \emph{a} or \emph{e}, depending on vowel harmony}) and for ``no possessor'' (modelled as the absence of the possessor morpheme), further processing is  needed (using a hack involving a \texttt{.t4x} file processed by 1st-level transfer without calling lexical transfer). 

The idea is that for these ``bare'' NPs, case and possesive markers need to be determined at a later time. This is needed for a number of constructs: these are only some examples:
\begin{itemize}
\item Possessive constructs: when NPs are pre-modified by a postpositional phrase with the genitive case postposition (\emph{-NI\c{n}}), the person and number of the SN in the postpositional phrase (called \(\mathrm{GenP}\)) will result in a possessive ending in the modified NP.
\item Translation of sentences with the main verb \emph{to have}, as the English subject NP has to be translated as a locative PP using the postposition \emph{-DA}. The object becomes the subject of the sentence and the main verb in Kazakh is the \emph{copula}:
\emph{The professor has a book} \(\to\) \emph{Professorda kitap bar} ('At the professor a book available [is]',  similar to the Russian construction); \emph{The professor had a book} \(\to\) \emph{Professorda kitab bar edi} ('At the professor a book available was').
\item Translation of obligative sentences with \emph{must} + verb or \emph{have to} + verb, as the English subject NP has to be translated as a genitive PP using the postposition \emph{-NI\c{n}} and the person and  number have to be transferred to the gerund of the verb. The sentence uses the adjective \emph{kerek} and the \emph{copula} (again, absent in present tense):
\emph{I have to go} \(\to\) \emph{Meni\c{n} barwim kerek -} ('My going [is] needed').
\item Translation of verbs such as \emph{need}: \emph{I need a book} \(\to\) {Ma\u{g}an kitap kerek -} ('To me a book needed [is]'); \emph{I needed a book} \(\to\) \emph{Ma\u{g}an kitap kerek edi} ('To me a book needed was')
\end{itemize}
We have generalized the idea of chunk-level tags to all chunks, both for those tags whose value has to be determined at the \texttt{.t2x} level and for those whose value has to be made available to \texttt{.t2x} rules; this makes \texttt{.t2x} rules more general and simpler to write.

\section{Translation of prepositional phrases}
On encountering an English prepositional phrase, there are three possible outcomes, and none of them seems to need further treatment, so the current rules will be kept more or less as they are, with the exception that the person and number of the main noun of \emph{genitive} PPs ending in the postposition \emph{-NI\c{n}} needs to be marked at the chunk level so that it may be used to form a possessive morpheme in the NP they modify. Using references is not necessary as they will not change. In addition, for all prepositional phrases, an unsolved possessive tag will be used to determine the possessive of its main noun in case there is another genitive phrase premodifying it.
Three cases may occur:
\begin{itemize}
\item The prepositional phrase results in a simple postpositional phrase using the locative \emph{-DA}, ablative \emph{-DAn}, etc., but not the genitive \emph{-NI\c{n}}:
\begin{center}
\Tree [.PP [.P in ] \qroof{the beautiful garden}.NP ] \(\to\) \Tree [.PP \qroof{ædemi baqša}.NP [.P -da ] ]
\end{center}
\item The prepositional phrase results in a simple postpositional phrase using the genitive \emph{-NI\c{n}}, which will be marked \(\mathrm{GenP}\):
\begin{center}
\Tree [.PP [.P of ]\qroof{the beautiful garden}.NP ] \(\to\) \Tree [.GenP \qroof{ædemi baqša}.NP [.P -ny\c{n} ] ]
\end{center}
\item The prepositional phrase results in a complex postpositional phrase based around a noun such as \emph{ast}, \emph{ust}, etc.:
\begin{center}
\Tree [.PP [.P under ] \qroof{the garden}.NP ] \(\to\) \Tree [.PP [.GenP [.PP \qroof{baqša}.NP -ny\c{n} ] \qroof{astyn}.NP ] [.P -da ] ]
\end{center}
\end{itemize}
Only in the second case there is need to assign the person and number of the noun to the whole chunk (phrase) for further processing, as it will reflect as a possessive in the NP it modifies.
\begin{center}
\Tree [.PP [.P of ] \qroof{the beautiful garden}.NP ]  \(\to\) \Tree [.PP(pers=3,num=sg) \qroof{ædemi baqša}.NP [.P -ny\c{n} ] ]
\end{center}
In this case, person and number can be considered \emph{solved} as there will be no need to change them; therefore, there is no need to use references (as it will be seen, this is necessary for ``bare'' NPs). 

\section{Translation of ``bare'' NPs}
When translating an English ``bare'' NP which is not processed as part
of any prepositional phrase (for instance, \emph{The professor} and
\emph{a big book} in \emph{The professor wrote a big book}, or
\emph{museum} in \emph{museum of modern art}), it is necessary to
leave its case and its possessor open for change (see the examples in
the introduction):
\begin{center}
\Tree [.NP [.D a ]  [.N book ] ] \(\to\) \Tree [.{NP(poss=?,case=?,pers=3,num=sg)} [.N kitap ] ]
\end{center}
In this case, the possessor and the case are open for the whole chunk and they have to be propagated to the noun via unsolved references to be solved at the 3rd (\texttt{.t3x}) level. Person and number can be considered solved, but need to be available at the chunk level for 2nd-level (\texttt{.t2x}) processing, for instance, so that verbs agree with their subjects. Note that in the current version of transfer, as found in the main branch (revision 45092, June 16, 2013), number and person are global variables set by NPs that are ``caught'' by verb phrases to determine the form of verbs. This has been abandoned and now agreement is treated in the 2nd-level transfer via references and propagated to lexical forms at the 3rd level (\texttt{.t3x}). This involves considerable work, as it makes some of the macros in 1st-level (\texttt{.t1x}) transfer useless, while other macros have to be defined (current rules contain extensive repetition of code generated through cut-and-paste which needs to be factored out into macros).

\section{Translation of VPs}

[A number of verb-phrase chunks have been written in \texttt{.t1x} during this period, but in the old (\emph{tagless}) format, and many have to be transformed to the new (\emph{tagged}) format (see section~\ref{se:todo}); examples: \emph{can} and \emph{could} constructs, tenses such as future continuous and future perfect, etc.]  



As said above, instead of the current method that just lets ``bare'' English NPs presumably acting as subjects to set global variables for person and number that will then be read by VP chunks to generate the appropriate forms, these ``bare'' NPs are now being marked at the chunk level with person and number tags (which would be constants) and  VPs are being marked at the chunk level with person and number to be determined, linked via references to the appropriate morphemes in the appropriate verb lexical forms. The chunk-level person and number to be determined are rewritten by the appropriate 2nd-level \texttt{.t2x} transfer rules, and are propagated to lexical forms at the 3rd-level transfer stage (\texttt{.t3x}).

Other indicators that have to be set at the chunk level are negation
(for negative verbs) and conditional (which will be handled as a
tense).

Negation can be easily determined at the \texttt{.t1x} level when the
English VP chunk contains \emph{not}, as in \emph{I don't play}
\(\to\) \emph{Men oynamaymwn}, but may need to be detemined at the
\texttt{.t2x} level in sentences having a non-negative VP but a
negative word like those starting with \emph{-eš}, like \emph{I write
  nothing} \(\to\) \emph{Men ešnærse \v{z}as\textbf{ba}ymwn}, which
requires a negative form of the verb. Negation is, as possession or
plural in noun phrases, an optional morpheme in the current
morphology, one, therefore, that has to be solved by the \texttt{.t4x}
hack described above.

Conditional constructions work by changing the form of the verb to a
special form. In the present, the resulting form can be seen as a
change in tense (\emph{I play} \(\to\) \emph{oynaymwn} =
\texttt{oyna.v.tv.aor.p1.sg}; \emph{If I play} \(\to\) \emph{oynasam}
= \texttt{oyna.v.tv.gna-cond.p1.sg}; \emph{If I don't play} \(\to\)
\emph{oynamasam} = \texttt{oyna.vblex.neg.gna-cond.p1.sg}). The past
seems to use an impersonal form (a verbal adjective, with no person or
nmumber agreement, in the locative case; \emph{oyna\u{g}anda} =
\texttt{oyna.v.tv.ger-past.loc}). Verbal nouns in other forms (such as
the ones needed to translate \emph{must} or \emph{have to} constructs,
which are realized as verbal nouns with possessives) make it necessary
for VPs to have a possessive tag to be solved later.

\section{Translation of adjective phrases}

Adjective phrases (\(\mathrm{AdjP}\)) are a bit simpler. Adjectives in
Kazakh do not show agreement, so AdjP can probably be generated as
chunks having no tags. However, the special case of the superlative
should perhaps be treated as a noun phrase with a missing (null) noun,
as it may show posessive morphemes as NPs do. The label \(\mathrm{AdjP}\)
has been temporarly changed here to something like \(\mathrm{SupP}\) as we were a bit hesitant on using just \(\mathrm{NP}\). For instance, the
translation of \emph{the most beautiful} is \emph{e\c{n} ædemi} (where
\emph{e\c{n}} is the superalative marker) but the translation of
\emph{the most beautiful of the books} is \emph{kitaptardy\c{n} e\c{n}
  ædemisi}, with a third-person possessive marking following the
adjective, as if a second \emph{kitap} had been deleted from a hypothetical \emph{kitaptardy\c{n} e\c{n} ædemi kitaby} ('the most
beautiful book of the books').\footnote{The current morphology marks
  some of these changes as \emph{substantivation}.} Comparatives do
not need special marking and may be treated as regular adjective
phrases where the adjective receives the \emph{-IRaK} (\emph{yraq}, \emph{irek}, etc.) comparative endings. Comparatives, as superlatives, are realized in English in two different ways (using synthetic \emph{-er}/\emph{-est} suffixes or prepending \emph{more}/\emph{most}): as a results, \emph{.t1x} rules are duplicated.

\begin{table}
\begin{center}
\begin{tabular}{c|p{3cm}|p{3cm}|p{3.5cm}}
\hline\hline
\textsc{Phrase} & \textsc{Description} & \textsc{Chunk-level tags} & \textsc{notes} \\\hline
NP & Noun phrase & number*, person, possessor*, case \\\hline
VP & Verb phrase & number, person, tense/conditionality, possessive*, negation* \\\hline
PP & Postpositional phrase (except \emph{-NI\c{n}}) & number, person, possessor*, and case & (number and person probably not necessary but parallel to GenP) \\\hline
GenP & Postpositional phrase with \emph{-NI\c{n}}) & number, person, possessor*, and case \\\hline
AdjP & Adjectival phrase (except superlatives with \emph{e\c{n}}) & (none) \\\hline
SupP & Superlative phrase & (adjectival phrase with \emph{e\c{n}} & possessor*, case (merging with NP is to be considered) \\\hline
\end{tabular}
\end{center}
\caption{Chunk-level tags currently associated to each type of chunk. Those marked with an asterisk correspond to optional morphemes that will need special treatment in a \texttt{.t4x} module (see text).}
\label{tb:chunks}
\end{table}

\section{Progress}
As said above, structural transfer rehaul is taking place in a branch
inside the main \texttt{apertium-eng-kaz} directory. All changes are
marked with the word `\texttt{NEW}'.

\subsection{June 19, 2013}

In \texttt{apertium-eng-kaz.eng-kaz.t1x}, only the noun phrase consisting of a single noun and the prepositional phrase consisting of preposition followed by a single noun have been changed. In \texttt{apertium-eng-kaz.eng-kaz.t2x}, a rule to translate \(\mathrm{NP}\;\mathrm{GenP}\to\mathrm{GenP}\;\mathrm{NP}\) which transfers the possessor from the \(\mathrm{GenP}\)
 of the \(\mathrm{NP}\) has been generated. Minor adjustments to the \texttt{.t4x} file have been performed.
 

\subsection{June 20, 2013}
\begin{description}
\item[\texttt{apertium-eng-kaz.eng-kaz.t2x}:] the noun phrase containing a single subject pronoun, the verb phrase containing a single verb have been transferred to the new model. Three more chunks have been defined, for superlative phrases (SupP) of the form \emph{the most beautiful}, of the form \emph{biggest}, and of the form \emph{the biggest}; the Kazakh superlative marker \emph{e\c{n}} has been added to dictionaries as a \emph{preadverb}.
\item[\texttt{apertium-eng-kaz.eng-kaz.t2x}:] rules have been added: 
  \begin{itemize}
  \item to generate the accusative case when translating \(\mathrm{NP}\) \(\mathrm{VP}\) \(\mathrm{NP}\) \(\to\) \(\mathrm{NP}\) \(\mathrm{NP}\) \(\mathrm{VP}\): \emph{I see streets} \(\to\) \emph{Men kö\v{s}eler\textbf{di} köremin}
  \item to test the generation of possessives on superlatives in \emph{the biggest of gardens} \(\to\) \emph{bak\v{s}salardy\c{n} e\c{n} ulken\textbf{i}} using the rule \(\mathrm{SupP}\) \(\mathrm{GenP}\) \(\to\) \(\mathrm{GenP}\)  \(\mathrm{SupP}\)
  \item to ensure person and number agreement in the sequence  \(\mathrm{NP}\)  \(\mathrm{VP}\) (through \texttt{.t2x} rules instead of global variables as before): \emph{We see} \(\to\) \emph{olar köre\textbf{miz}}.
  \end{itemize}
As before, minor adjustments have been made to the \texttt{.t4x} file. 



\end{description}
\subsection{June 21, 2013}

The conversion of rules from the old (\emph{tagless chunks}) format to the new (\emph{heavily tagged}) format in \emph{.t1x} has not been completed (rules for subject pronouns, determiner--noun and preposition--determiner--noun have been converted to the new format, and some specific chunks for verbs such as \emph{must} and \emph{have} have been added), but some ``example'' \texttt{.t2x} rules have been added to deal with many of them:
\begin{itemize}
\item A rule to deal with  \(\mathrm{PP}\) \(\mathrm{GenP}\) \(\mathrm{GenP}\) \(\mathrm{PP}\) \(\to\) \(\mathrm{GenP}\) \(\mathrm{GenP}\) \(\mathrm{PP}\) so that, according to Kazakh grammar, the first \(\mathrm{GenP}\) does not get the \emph{-NI\c{n}} ending but is instead in attributive case.
\item A rule to get the accusative case for the object in sentences of the form \(\mathrm{NP}\) \(\mathrm{VP}\) \(\mathrm{PP}\) \(\to\) \(\mathrm{NP}\mathtt{.nom}\) \(\mathrm{NP}\mathtt{.acc}\) \(\mathrm{VP}\) (does not check for transitivity)
\item Affirmative \(\mathrm{NP}\) \emph{have} \(\mathrm{NP}\) \(\to\) \(\mathrm{NP}\mathtt{.loc}\) \(\mathrm{NP}\mathtt{.nom}\) \emph{bar} \emph{copula} (with \emph{have} as a lexical verb) constructions for the present and the past.
\item Affirmative \emph{must} construction \(\mathrm{NP}\) \emph{must} \emph{infinitive} \(\to\) \(\mathrm{NP}\mathtt{.gen}\) \emph{verbal-noun} \emph{kerek} \emph{coupla}
\end{itemize}
All of these constructs should be extended to deal with objects, modifiers, etc., by writing long \texttt{.t2x} rules. Currently, \texttt{.t2x} is \emph{almost} ready.
As usual, some \emph{.t4x} adjustments were necessary. Specific phrases such as \(\mathrm{VP\_have}\)
\(\mathrm{VP\_must\_inf}\) etc have been created to make \texttt{.t2x} rules easier to write.

\subsection{June 22, 2013}
The conversion of rules from the old (\emph{tagless chunks} to the new
\emph{tagged chunk} format for noun-phrase (NP) rules in \texttt{.t1x}
has been completed (10 more rules). VP rules for \emph{can} and
\emph{should} have been added. A rule for bare adjective phrases has
also been added.

\section{Work to do}
\label{se:todo}
Table~\ref{tb:todo} describes work that should ideally be completed before August 31. The list is incomplete. Any new rules should be accompanied by appropriate tests for regression testing.
\begin{table}
  \centering
  \begin{tabular}{r|p{7.5cm}|p{1.7cm}|p{2.2cm}}
    \hline\hline
    \textsc{} & \textsc{Task} & \textsc{Proposed deadline} & \textsc{Who} \\
    \hline
    1 & Normalize superblank management everywhere so that we get valid XML/HTML from valid XML/HTML & 21/07/2013 & Mikel \\\hline
    2 & Try to generalize VP treatment in \texttt{.t2x} and \texttt{.t3x} rule files so that we can use less interchunk \texttt{.t2x} rules. This may involve deferring some processing to \texttt{.t3x} (we need to learn what is done in non-trivial \texttt{.t3x} rule files in other Apertium language pairs & 21/07/2013 & Mikel \\\hline
    3 & Check the \emph{fake lemmas} of \texttt{.t1x} chunks so that they actually describe their internals (convenient for management purposes) & 29/06/2013 & Mikel \\\hline
    4 & Transform all VP chunks from the \emph{tagless} to the \emph{tagged} format & 05/07/2013 & Aida, Aizhan, Assem (supervised by Mikel) \\\hline
    5 & Build \texttt{.t1x} and \texttt{.t2x} rules to deal with the \emph{need} construct (\emph{I need[ed] a book} \(\to\) \emph{Ma\u{g}an kitap [edi]}) & 05/07/2013 & Aida (supervised by Mikel) \\\hline
    6 & Generate negative forms of \emph{to be} & 05/07/2013 & Aida, Aizhan, Assem (supervised by Mikel) \\\hline
    7 & Generate \texttt{.t2x} rules for the verb \emph{to be} (copula) & 12/07/2013 & Aida, Aizhan, Assem (supervised by Mikel) \\\hline 
   8 & First check against regression testing, and merge the development branch into the main branch & 20/07/2013 & Mikel \\
   9 & Comparatives: first, generate \texttt{.t1x} rules for AdjP containing comparatives; then write \emph{.t2x} rules for \emph{than} (\emph{Bigger than the book} \(\to\) \emph{kitap\textbf{qa} \textbf{kara\u{g}anda} ulken\textbf{irek}}), where the noun gets the dative case & 20/08/2013 & Aida, Aizhan, Assem (supervised by Mikel) \\\hline
    10  & Generating negative forms of verbs from negative (\emph{e\v{s}-}) words (\texttt{.t2x} work) & 20/08/2013 & Aida, Aizhan, Assem (supervised by Mikel) \\\hline
   11  & Adding vocabulary & Continuous task & Aida, Aizhan, Assem (supervised by Mikel) \\\hline
   12 & Second check against regression testing & 29/08/2013 & Mikel \\\hline
  \end{tabular}
  \caption{Work to be completed in Summer 2013}
  \label{tb:todo}
\end{table}

Other items that could be added to the work this summer but that are not scheduled in Table~\ref{tb:todo}
\begin{itemize}
\item Interrogative sentences: yes/no questions (marked with \emph{BA}, which, even if written as a separate word, shows vowel and consonant harmony), informative questions (Kazakh does not move \emph{wh-}words, which are in their canonical positions), tag questions.
\item Noun-noun compounds (which form structures in Kazakh which are not marked by the genitive postposition \emph{-NI\c{n}} but show possessives: \emph{Qazaqstan president\textbf{i}} instead of \emph{Qazakhstanni\c{n} presidenti}).
\item Passive voice (partial support is already there)
\item Taking care of capitals by means of \texttt{get-case-from} 
\end{itemize}
\end{document}
